This research has been intriguing and interesting but has fallen victim to many shortcomings that come with textual data and human emotions.
There just might be a clear correlation between a President's tone on a specific topic in the United States and their political party but the results found here cannot prove such a thing.
The art of converting qualitative text data into actionable quantitative data is still a process in its infancy and many advancements are to come in this field before it flourishes into a more accurate and effective prediction method.

\section{Complications}
The complications arose mostly from the text data and manipulating it effectively to translate it into numbers while retaining as much meaning and context as possible.
There is only so much meaning and interpretation that can be derived from purely the text without consideration for the socio-political climate at the time that the speech was given, which is a much harder problem to solve and quantify.
The potential for this research to aid in political science research on presidential profiles is high, but as a stand-alone method for interpreting Presidential party alignment it needs more work and fine-tuning to do that effectively.

Another major pitfall that this research ran into was not having a large enough data source to compile specific profiles for each president to form their political profiles in stronger ways to shape a political party position.
The scope of the dataset was limited to State of the Union Addresses as they are consistently delivered each year by the president so the standards were understood and known for Presidents past and future.
This consistency is important since the speeches can be interpreted and analyzed given the same basic list of information to look for in this address.
Also the typical fashion in which the State of the Union is delivered mandates the President address each major topic of interest concerning the United States, thus lending itself to be analyzed in this automatic fashion.
A possibly more effective, yet time-consuming approach, would be to include personal writings and other speeches given by the president and discern them for meaning and add them to the corpus of text data analyzed.
Some of these documents would be short and some of them would need to be manually tagged for meaning depending on what the content of the speech was, but perhaps this would provide greater insight into the Presidential profile and thus create a stronger party profile on which to predict Presidential alignment.

Whether the shortcoming of these predictions come from a lack of data or a lack of correlation is impossible to tell and no such conclusion can be made at this time.
Perhaps, their tone when speaking on certain topics can show their party alliance but no strong evidence has been found thus far.
And there could never be an accurate gauge that is reliable enough to predict Presidential party given the nature of how a President gets elected in the first place.
Most Presidents are moderate enough where they can swing at least a portion of the vote in their favor.
So, although a president might have particularly strong feelings on some categories they have more moderate opinions on others that average out to a moderate take on many things.
This inherently moderate nature of the President doesn't bode well for predicting their party alignments but with diversified text data this could potentially be rectified.
This research could also be reproduced using the Supreme Court decisions as text data and predict party alliance based on how the Supreme Court Judges decide since their political alignment is better known and can be pinpointed more resolutely than the President's since they decide on every case and have more consistent output of text data to analyze.

\section{Future Work}
There is much that could be done to continue this research to make it more effective, and also to make it more interesting and intriguing to look at.
Some important future work would be adding additional visualizations that further breaks down all of the data discussed here.
There is a great amount of it and expanding upon these visualizations would make it much more effective to look at and analyze.
These visualizations would incorporate historical events and allow for tracking of a president's tone over time as it correlate to major historical events in the United States, as well as the world.
These visualizations could also include a per year approach that allows a user to look at a particular year to see the president and sentiment score as well as other important economic and social information to examine the correlations between the well-being of the nation and the overall attitude of the President.

Another intriguing avenue to pursue would be to look at House and Senate majorities versus ruling presidency and how many bills were passed and how many laws were implemented, and compare that to the tone of the President.
Perhaps to see if the frustrations of getting bills and laws rejected would reflect itself in a more negative tone of the President.
An interesting expansion to this research that might warrant a whole new thesis itself would be exploring the Supreme Court Decisions and crafting party alignment using the Supreme Court Justices' decisions and statements and attempting to use that to predict a person's political party alignment based on the content of their speeches or writings.

\section{Final Thoughts}
This research has been intriguing and rewarding and provided quite a lot of obstacles and challenges.
There is still much to explore in Natural Language Processing and Machine Learning as the surface was only scratched throughout this research.
The overall question of Presidential tone and political party alignment still remains to be explored and hopefully this helps as a starting point for future research into this immensely interesting avenue of research.